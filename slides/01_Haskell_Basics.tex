\documentclass{beamer}

\usetheme{metropolis}

\usepackage[ngerman]{babel}
\usepackage[autostyle=true,german=quotes]{csquotes}
\usepackage[linewidth=1pt]{mdframed}
\usepackage{hyperref}
\usepackage{makecell}
\usepackage{pifont}
\usepackage{tikz}
\usetikzlibrary{positioning, calc, arrows, fit, decorations.pathreplacing, shapes, shapes.multipart, snakes}
\usepackage{verbatim}
\usepackage{textcomp}
\usepackage{centernot}
\usepackage{tabularx}
\usepackage{ulem}
\usepackage{mathpartir}
%\usepackage{pdfpages}

\batchmode

\hypersetup{
	colorlinks,
	urlcolor=blue,
	linkcolor=black % for ToC
}
\newenvironment{qaa}[1]{
	#1

	\begin{mdframed}
		\small
}{
	\end{mdframed}
}

\newcommand{\true}{\ding{51}}
\newcommand{\false}{\ding{55}}
\newcommand{\code}[1]{
	\begin{mdframed}
		\verbatiminput{#1}
	\end{mdframed}
}

% lambda calculus stuff

\newcommand{\app}[2]{#1\;#2}
\newcommand{\lam}[2]{\lambda #1.\, #2}
\newcommand{\appp}[3]{\app{\app{#1}{#2}}{#3}}

\newcommand{\ctrue}{c_\text{true}}
\newcommand{\cfalse}{c_\text{false}}
\newcommand{\cand}{\text{AND}}

\newcommand{\csucc}{\text{succ}}
\newcommand{\cn}[1]{c_{#1}}

\newcommand{\cpair}{\text{pair}}
\newcommand{\cfst}{\text{fst}}
\newcommand{\csnd}{\text{snd}}
\newcommand{\ccurry}{\text{curry}}
\newcommand{\cuncurry}{\text{uncurry}}

\newcommand{\cnil}{\text{nil}}
\newcommand{\ccons}{\text{cons}}
\newcommand{\chead}{\text{head}}
\newcommand{\ctail}{\text{tail}}
\newcommand{\creplicate}{\text{replicate}}

\newcommand{\subst}[3]{(#1)\left[#2\,\to\,#3\right]}


\title{Tutorium 01: Haskell Basics}
% \subtitle{}
\author{Paul Brinkmeier}
\institute{Tutorium Programmierparadigmen am KIT}
\date{26. Oktober 2021}

\begin{document}

\begin{frame}
        \titlepage
\end{frame}

\section{Organisatorisches}

\begin{frame}{Organisatorisches}
        \begin{itemize}
                \item \texttt{pp-tut@pbrinkmeier.de}
                \begin{itemize}
                        \item Für Feedback und Fragen
                \end{itemize}
                \item \url{https://github.com/pbrinkmeier/pp-tut}
                \begin{itemize}
                        \item Folien
                        \item Codebeispiele
                        \item Liste von Klausuraufgaben
                \end{itemize}
                \item Bitte Laptop o.Ä. mitbringen
                \begin{itemize}
                    \item Für Mitarbeit im Tutorium: \url{https://codi.pbrinkmeier.de/s/wq88qbGYy}
                    \item ... und natürlich zum selber Rumspielen
                \end{itemize}
        \end{itemize}
\end{frame}

\begin{frame}{Übungsbetrieb}
        \begin{itemize}
                \item ProPa hat keinen Übungsschein
                \item $\leadsto$ ÜBs zur eigenen Übung!
                \item Abgabe:
                \begin{itemize}
                  \item \url{https://praktomat.cs.kit.edu/pp_2021_WS/tasks}
                  \item Kasten im Infobau-UG
                  \item Wenn ihr die Abgabe verpasst habt auch per Mail
                \end{itemize}
        \end{itemize}
\end{frame}

\begin{frame}{Klausur}
        \begin{itemize}
                \item Termin: ??.??.2022, normalerweise April/Mai
                \item Papier-Materialien dürfen mitgebracht werden!
                \item $\leadsto$ Skript, Mitschriebe, \enquote{Formelsammlung}
        \end{itemize}
\end{frame}

% TODO: Personal section

\section{Heutiges Programm}
\begin{frame}{Programm}
        \begin{itemize}
                \item Haskell installieren
                \item Wiederholung der Vorlesung
                \item Aufgaben zu Haskell
        \end{itemize}
\end{frame}


\section{Haskell}

\begin{frame}{Externe Ressourcen}
  \begin{itemize}
    \item Learn You a Haskell (\href{http://learnyouahaskell.com}{learnyouahaskell.com})
    \begin{itemize}
      \item Vorlesungsstoff $\subseteq$ erste zehn Kapitel
      \item Zehntes Kapitel enthält gut erklärtes \enquote{Mini-Projekt}
    \end{itemize}
    \item 99 Haskell Problems (\href{https://wiki.haskell.org/H-99:_Ninety-Nine_Haskell_Problems}{wiki.haskell.org})
    \begin{itemize}
      \item Sammlung von Aufgaben mit Lösung
      \item Großer Teil zu Listen
    \end{itemize}
    \item Hoogle (\href{https://hoogle.haskell.org}{hoogle.haskell.org}): Dokumentation
  \end{itemize}
\end{frame}

\begin{frame}[fragile]{GHCi}
        \code{code/ghci.output}

        \begin{itemize}
                \item Von der VL verwendeter Haskell-Compiler: GHC
                \item Interaktive Haskell-Shell: \texttt{ghci}
                \item Installation:
                \begin{itemize}
                  \item \sout{Windows: Installer von Haskell-Website}
                  \item \sout{Linux: Je nach Distro \texttt{haskell-platform} oder \texttt{ghc} installieren}
                  \item \sout{macOS: \texttt{ghcup}}
                  \item Verwendet am besten \texttt{stack} (\href{https://www.haskellstack.org}{www.haskellstack.org})
                \end{itemize}
        \end{itemize}
\end{frame}

\begin{frame}{Module}
        \code{../demos/Maths.hs}

        \begin{itemize}
                \item Ein Haskell-Programm ist eine Folge von Funktionsdefinitionen.
                \item Funktionen müssen keine Argumente haben.
        \end{itemize}
\end{frame}

\begin{frame}{REPL}
        \code{code/ghci-maths.output}

        \begin{itemize}
                \item \texttt{ghci} ist ein sog. \enquote{Read-Eval-Print-Loop}
                \item \texttt{:l} --- Modul aus Datei laden
                \item \texttt{:r} --- Modul neu laden
                \item \texttt{:t} --- Typ eines Audrucks abfragen
        \end{itemize}
\end{frame}

\begin{frame}{Funktionen}
        \code{../demos/Maths.hs}

        \begin{itemize}
                \item Unterschied zu C-ähnlichen Sprachen: Keine Klammern/Kommata, \texttt{=}
                \item Leerzeichen als Syntax für \enquote{Funktionsaufruf}
        \end{itemize}
\end{frame}

\begin{frame}{Basistypen}
        \begin{tabular}{ l | c | c }
                Wert & Typ in Java & Typ in Haskell \\
                \hline
                \texttt{"Hello, World!"} & \texttt{String} & \pause \texttt{String} \\
                \texttt{'x'} & \texttt{char} & \pause \texttt{Char} \\
                5 & \texttt{int} & \pause \texttt{Int} \\
                9999999999999999999999999 & \texttt{BigInteger} & \pause \texttt{Integer} \\
                3.1415927 & \texttt{float} & \pause \texttt{Float} \\
                3.141592653589793 & \texttt{double} & \pause \texttt{Double} \\
                \texttt{[Tt]rue}, \texttt{[Ff]alse} & \texttt{boolean} & \pause \texttt{Bool} \\
        \end{tabular}
\end{frame}

\begin{frame}{Basistypen}
        \code{../demos/Maths.hs}

        \begin{itemize}
                \item Schreibt ein Modul \texttt{FirstSteps} mit folgenden Funktionen:
                \begin{itemize}
                        \item \texttt{double x} --- Verdoppelt \texttt{x}
                        \item \texttt{dSum x y} --- Verdoppelt \texttt{x} und \texttt{y} und summiert die Ergebnisse
                        \item \texttt{area r} --- Fläche eines Kreises mit Radius \texttt{r}
                        \item \texttt{sum3 a b c} --- Summiert \texttt{a}, \texttt{b} und \texttt{c}
                        \item \texttt{sum4 a b c d} --- Summiert \texttt{a}, \texttt{b}, \texttt{c} und \texttt{d}
                \end{itemize}
        \end{itemize}
\end{frame}

\begin{frame}{Listen}
        \begin{itemize}
                \item \texttt{sum3}, \texttt{sum4}, \texttt{sumX} zu schreiben ist irgendwie doof
                \pause
                \item Lösung des Problems: Listen
                \item \texttt{[a]} ist der Typ einer Liste, deren Elemente von Typ \texttt{a} sind
                \item $\leadsto$ Listen sind homogen, nur eine Art von Element
        \end{itemize}

        \code{../demos/Lists.hs}

        \begin{itemize}
                \item Ein Ausdruck des Typs \texttt{[a]} hat genau einen von zwei Werten:
                \begin{itemize}
                        \item \texttt{[]} --- die leere Liste
                        \item \texttt{(h : t)} --- Element + Rest, mit \texttt{h :: a} und \texttt{t :: [a]}
                \end{itemize}
        \end{itemize}
\end{frame}

\begin{frame}{Funktionstypen}
        \begin{itemize}
                \item \textbf{Funktionen sind Werte}
                \item $\leadsto$ Funktionen haben einen Typ
                \item Allgemeine Form: \texttt{x -> y}
                \item Beispiel: \texttt{length :: [a] -> Int}
                \begin{itemize}
                        \item Java: \texttt{Function<List<A>, Integer> length;}
                        \item C: \texttt{int (*strlen)(char *str)};
                \end{itemize}
        \end{itemize}
\end{frame}

\begin{frame}{Funktionstypen, mehrere Argumente}
        \code{../demos/Maths.hs}

        \begin{itemize}
                \item Funktionen sind vom Typ \texttt{x -> y}
                \item Welchen Typ hat dann \texttt{add}? \\
                \pause
                $\leadsto$ \texttt{Num a => a -> a -> a}
                \item \texttt{->} ist rechtsassoz. \\
                $\leadsto$ \texttt{a -> a -> a} $\equiv$ \texttt{a -> (a -> a)}
        \end{itemize}
\end{frame}

\begin{frame}{Unterversorgung}
        \begin{itemize}
                \item Haskell-Funktionen sind \enquote{ge-Curry-d}
                \item D.h.: Jede Funktion hat exakt ein Argument
                \item Funktionen mit mehreren Argumenten geben solange Funktionen zurück, bis sie ausreichend \enquote{versorgt} sind
        \end{itemize}

        \code{code/curry.txt}
\end{frame}

\begin{frame}{Fallunterscheidung: if-then-else}
        \code{../demos/MaxIf.hs}

        \begin{itemize}
                \item Einfachste Form der Fallunterscheidung
                \item \texttt{if <Bedingung> then <WertA> else <WertB>}
                \pause
                \item Das ist nichts anderes als der ternäre Operator in C-ähnlichen Sprachen:
                \begin{itemize}
                        \item \texttt{<Bedingung> ? <WertA> : <WertB>}
                \end{itemize}
        \end{itemize}
\end{frame}

\begin{frame}{Fallunterscheidung: Guard-Notation}
        \code{../demos/MaxGuard.hs}

        \begin{itemize}
                \item \enquote{Guard}-Notation
                \item Wird einfach von oben nach unten abgearbeitet
                \item Oft kürzer als \texttt{if a then x else if b then y else z}
                \pause
                \item \texttt{otherwise == True}
        \end{itemize}
\end{frame}

\begin{frame}{Fallunterscheidung: Pattern Matching}
        \code{../demos/Bool.hs}

        \begin{itemize}
                \item Statt Variablen einfach Werte in den Funktionskopf setzen
                \item Mehrere Funktionsdefinitionen möglich
                \item Funktioniert nicht immer (bspw. bei \texttt{max})
                \item Hier nützlich: \texttt{\textunderscore} \enquote{ignoriert} Argument
        \end{itemize}
\end{frame}

\begin{frame}{Aufgabe: Summen}
  \code{../demos/Series.hs}

  \texttt{squareSum xs} berechnet $\sum\limits_{x \in \texttt{xs}} x^2$.
  Schreibt folgende Funktionen:

  \begin{itemize}
    \item \texttt{cubeSum xs}: $\sum\limits_{x \in \texttt{xs}} x^3$
    \item \texttt{mysterySum xs}: $\sum\limits_{x \in \texttt{xs}} \frac{1}{(4x + 1)(4x + 3)}$ (wofür ist das gut?)
  \end{itemize}

  Beispiele zum Testen:

  \begin{itemize}
    \item \texttt{cubeSum [0..10] = 3025}
    \item \texttt{mysterySum [0..10] = 0.38702019080795513}
  \end{itemize}
\end{frame}

\begin{frame}{Aufgabe: Summen mit Funktionen höherer Ordnung}
  \begin{itemize}
    \item \texttt{squareSum xs}: $\sum\limits_{x \in \texttt{xs}} x^2$
    \item \texttt{cubeSum xs}: $\sum\limits_{x \in \texttt{xs}} x^3$
    \item \texttt{mysterySum xs}: $\sum\limits_{x \in \texttt{xs}} \frac{1}{(4x + 1)(4x + 3)}$ (Konvergiert gg. $\frac{\pi}{8}$)
  \end{itemize}

  Hier gibt es ein gemeinsames Muster:
  \pause
  $$
  \sum\limits_{x \in \texttt{xs}} \texttt{f}(x)
  $$

  Schreibt eine Funktion \texttt{funcSum f xs}, die dieses Muster umsetzt. Schreibt damit neue Versionen von \texttt{squareSum}, \texttt{cubeSum} und \texttt{mysterySum}!\\
  Beispiel: \texttt{funcSum (\textbackslash{}x -> x) [0..10] = 55}
\end{frame}

\begin{frame}{Cheatsheet: Listen}
        \begin{itemize}
          \item \texttt{[]}, \texttt{(:)}
          \item \texttt{(++) :: [a] -> [a] -> [a]}
          \item \texttt{head :: [a] -> a}
          \item \texttt{tail :: [a] -> [a]}
          \item \texttt{null :: [a] -> Bool}
          \item \texttt{length :: [a] -> Int}
          \item \texttt{isIn :: [a] -> a -> Bool}
          \item \texttt{elem :: a -> [a] -> Bool}
          \item \texttt{minimum, maximum :: Ord a => [a] -> a}
          \item \texttt{reverse :: [a] -> [a]}
          \item \texttt{take, drop :: Int -> [a] -> [a]}
          \item \emph{Endrekursion}, \emph{Akkumulatortechnik}, \emph{List comprehension}
        \end{itemize}
\end{frame}

\begin{frame}{Cheatsheet: Basics}
        \begin{itemize}
          \item \texttt{(==) :: Eq a => a -> a -> Bool}
          \item \texttt{(<), (<=), (>), (>=) :: Ord a => a -> a -> Bool}
          \item \texttt{min, max :: Ord a => a -> a -> a}
          \item \texttt{type String = [Char]}
          \item Syntax:
          \begin{itemize}
            \item \emph{\texttt{if ... then ... else}}
            \item \emph{\texttt{case ... of { ... }}}
            \item \emph{Guard-Notation}, \emph{Pattern-Matching}
            \item \emph{Lambda-Notation}
            \item \emph{\texttt{where}} vs. \emph{\texttt{let}}
          \end{itemize}
          \item \emph{Anonyme Funktionen}
        \end{itemize}
\end{frame}

\begin{frame}{Cheatsheet: Funktionen höherer Ordnung}
        \begin{itemize}
          \item \emph{Currying}, \emph{Unterversorgung}
          \item \emph{$\lambda$-Abstraktion}, \emph{gebundene/freie Variablen}
          \item \texttt{(.), comp :: (b -> c) -> (a -> b) -> a -> c}
          \item \texttt{iter :: (t -> t) -> Integer -> (t -> t)}
          \item \emph{Funktionen sind Werte}
        \end{itemize}
\end{frame}

\begin{frame}{Aufgaben}
        Schreibt ein Modul \texttt{Tut01} mit:

        \begin{itemize}
                \item \texttt{fac n} --- Berechnet Fakultät von \texttt{n}
                \item \texttt{fib n} --- Berechnet \texttt{n}-te Fibonacci-Zahl
                \item \texttt{fibs n} --- Liste der ersten \texttt{n} Fibonacci-Zahlen
                \item \texttt{fibsTo n} --- Liste der Fibonacci-Zahlen bis \texttt{n}
                \item \texttt{productL l} --- Berechnet das Produkt aller Einträge von \texttt{l}
                \item \texttt{odds l} --- Ungerade Zahlen in \texttt{l}
                \item \texttt{evens l} --- Gerade Zahlen in \texttt{l}
                \item \texttt{squares l} --- Liste der Quadrate aller Einträge von \texttt{l}
        \end{itemize}
\end{frame}

\begin{frame}{Digits}
  Schreibt ein Modul \texttt{Digits} mit:

  \begin{itemize}
    \item \texttt{digits :: Int -> [Int]} --- Liste der Stellen einer positiven Zahl.
  \end{itemize}

  Bspw.:

  \code{code/digits.txt}
\end{frame}

\begin{frame}{Tipps für Blatt 1}
  \begin{itemize}
    \item Überlegt euch was die Funktionen ausgeben müssen, bspw.\\
      \texttt{pow1 9 3 == pow2 9 3 == pow3 9 3 == 729},\\ \texttt{root 2 100 == 10}, ...
    \item \texttt{error "Nachricht"} ist nützlich für Fehlerfälle
    \item Für \texttt{root}: Visualisiert die Intervallhalbierung auf Papier
    \item Für \texttt{isPrime}: Hier ist List comprehension praktisch
    \item Für \texttt{mergeSort}: Alle Basisfälle abdecken! Bspw.\\
      \texttt{mergeSort [] == []}, \texttt{mergeSort [42] == ?}
  \end{itemize}
\end{frame}

\end{document}
