\documentclass{beamer}
\usetheme{metropolis}

\usetheme{metropolis}

\usepackage[ngerman]{babel}
\usepackage[autostyle=true,german=quotes]{csquotes}
\usepackage[linewidth=1pt]{mdframed}
\usepackage{hyperref}
\usepackage{makecell}
\usepackage{pifont}
\usepackage{tikz}
\usetikzlibrary{positioning, calc, arrows, fit, decorations.pathreplacing, shapes, shapes.multipart, snakes, trees, automata}
\usepackage{verbatim}
\usepackage{textcomp}
\usepackage{centernot}
\usepackage{tabularx}
\usepackage[normalem]{ulem}
\usepackage{mathpartir}
\usepackage{pdfpages}
\usepackage{wasysym}
\usepackage{qtree}
\usepackage{pgfplots}

\batchmode

\hypersetup{
	colorlinks,
	urlcolor=blue,
	linkcolor=black % for ToC
}
\newenvironment{qaa}[1]{
	#1

	\begin{mdframed}
		\small
}{
	\end{mdframed}
}

\newcommand{\true}{\ding{51}}
\newcommand{\false}{\ding{55}}
\newcommand{\code}[1]{
	\begin{mdframed}
		\verbatiminput{#1}
	\end{mdframed}
}

% lambda calculus stuff

\newcommand{\app}[2]{#1\;#2}
\newcommand{\appr}[2]{\app{\uline{#1}}{\uwave{#2}}}
\newcommand{\lam}[2]{\lambda #1.\, #2}
\newcommand{\appp}[3]{\app{\app{#1}{#2}}{#3}}
\newcommand{\apppp}[4]{\app{\app{\app{#1}{#2}}{#3}}{#4}}
\newcommand{\apppr}[3]{\app{\appr{#1}{#2}}{#3}}

\newcommand{\lamlet}[3]{\texttt{let} \;\; #1 = #2 \;\; \texttt{in} \;\; #3}

\newcommand{\ctrue}{c_\text{true}}
\newcommand{\cfalse}{c_\text{false}}
\newcommand{\cand}{\text{AND}}

\newcommand{\csucc}{\text{succ}}
\newcommand{\cn}[1]{c_{#1}}

\newcommand{\cpair}{\text{pair}}
\newcommand{\cfst}{\text{fst}}
\newcommand{\csnd}{\text{snd}}
\newcommand{\ccurry}{\text{curry}}
\newcommand{\cuncurry}{\text{uncurry}}

\newcommand{\cnil}{\text{nil}}
\newcommand{\ccons}{\text{cons}}
\newcommand{\chead}{\text{head}}
\newcommand{\ctail}{\text{tail}}
\newcommand{\creplicate}{\text{replicate}}

\newcommand{\subst}[3]{(#1)\left[#2\,\to\,#3\right]}

% unification etc.

\newcommand{\functor}[2]{\texttt{#1}(#2)}
\newcommand{\pa}[1]{\texttt{#1}}
\newcommand{\plsunify}{\stackrel{!}{=}}
\newcommand{\unifier}[1]{\left[#1\right]}
\newcommand{\su}[2]{#1 \; \text{\pointer} \; #2}


\title{Tutorium 02: Mehr Haskell}
% \subtitle{}
\author{Paul Brinkmeier}
\institute{Tutorium Programmierparadigmen am KIT}
\date{08. November 2022}

\begin{document}

\begin{frame}
	\titlepage
\end{frame}

\begin{frame}{Notenskala}
	\begin{itemize}
		\item -
		\item Richtig, kleine Fehler
		\item Aufgabe nicht verstanden
		\item Grundansatz falsch
		\item Richtig!
		\item Richtiger Ansatz, aber unvollständig
	\end{itemize}
\end{frame}

\section{Heutiges Programm}
\begin{frame}{Programm}
	\begin{itemize}
		\item Übungsblatt 1
		\item Wiederholung der Vorlesung
		\item Aufgaben zu Haskell
	\end{itemize}
\end{frame}

\section{Übungsblatt 1}

\begin{frame}{Basisfälle und Spezialfälle}
	\code{code/pow3.txt}

	\begin{itemize}
		\item Spezial- und Fehlerfälle in den \enquote{Rumpf}
		\item Basisfälle für Iteration in die Endrekursion\\
			  $\leadsto$ Vermeidung von Codedopplung
	\end{itemize}
\end{frame}

\begin{frame}{Endrekursion}
	\code{code/pow3acc.txt}
\end{frame}

\begin{frame}{Sortierung}
	\begin{align*}
		\texttt{sort} \left[ 5, 2_1, 2_2, 2_3, 1 \right]
		&= \left[ 1, 2_1, 2_2, 2_3, 5 \right] \text{\true} \\
		&= \left[ 1, 2_3, 2_1, 2_2, 5 \right] \text{\false}
	\end{align*}

	\begin{itemize}
		\item Bei Aufgabe 2 war implizit gefordert, dass die Sortierfunktionen stabil sind.
		\item Originalreihenfolge soll nicht geändert werden.
		\item $\implies$ Bei \texttt{merge} nicht tauschen, \texttt{>=} vs. \texttt{>}, ...
	\end{itemize}
\end{frame}

\section{Wiederholung: Funktionen}

\begin{frame}{Cheatsheet: Listenkombinatoren}
  \begin{itemize}
    \item \texttt{foldr :: (a -> b -> b) -> b -> [a] -> b}
    \item \texttt{foldl :: (b -> a -> b) -> b -> [a] -> b}
    \item \texttt{map :: (a -> b) -> [a] -> [b]}
    \item \texttt{filter :: (a -> Bool) -> [a] -> [a]}
    \item \texttt{zipWith :: (a -> b -> c) -> [a] -> [b] -> [c]}
    \item \texttt{zip :: [a] -> [b] -> [(a, b)]}
    \item \texttt{and, or :: [Bool] -> Bool}
  \end{itemize}

  Idee: Statt Rekursion selbst zu formulieren verwenden wir fertige \enquote{Bausteine}, sogenannte \enquote{Kombinatoren}.
\end{frame}

\begin{frame}{Cheatsheet: Tupel und Konzepte}
  \begin{itemize}
    \item \emph{List comprehension}, \emph{Laziness}
    \item \texttt{[f x | x <- xs, p x]} $\equiv$ \texttt{map f (filter p xs)}\\
      Bspw.: \texttt{[x * x | x <- [1..]]} $\Rightarrow$ \texttt{[1,4,9,16,25,...]}
    \item \emph{Tupel}
    \item \texttt{(,) :: a -> b -> (a, b)} (\enquote{Tupel-Konstruktor})
    \item \texttt{fst :: (a, b) -> a}
    \item \texttt{snd :: (a, b) -> b}
  \end{itemize}
\end{frame}

\begin{frame}{Cheatsheet: Typen}
  \begin{itemize}
    \item \texttt{Char}, \texttt{Int}, \texttt{Integer}, ...
    \item \texttt{String}
    \item \emph{Typvariablen} \texttt{a}, \texttt{b}/\emph{Polymorphe Typen}:
    \begin{itemize}
      \item \texttt{(a, b)}: Tupel
      \item \texttt{[a]}: Listen
      \item \texttt{a -> b}: Funktionen
      \item Vgl. Java: \texttt{List<A>}, \texttt{Function<A, B>}
    \end{itemize}
    \item \emph{Typsynonyme}: \texttt{type String = [Char]}
  \end{itemize}
\end{frame}

\begin{frame}{Aufgaben}
	Schreibt ein Modul \texttt{Tut02} mit:

	\begin{itemize}
		\item \texttt{import Prelude hiding (map, filter, scanl, zip, zipWith)}
		\item \texttt{map} --- Einmal von Hand, einmal per Fold
		\item \texttt{filter} --- Einmal von Hand, einmal per Fold
		\item \texttt{squares l} --- Liste der Quadrate der Elemente von \texttt{l}
		\item \texttt{zip as bs} --- Erstellt Tupel der Elemente von \texttt{as} und \texttt{bs}
		\item \texttt{zipWith f as bs} --- Wendet komponentenweise \texttt{f} auf die Elemente von \texttt{as} und \texttt{bs} an
		\begin{itemize}
			\item Bspw. \texttt{zipWith (+) [1, 1, 2, 3] [1, 2, 3, 5] == [2, 3, 5, 8]}
		\end{itemize}
   		\pause
		\item \texttt{foldl}
		\item \texttt{scanl f i l} --- Wie \texttt{foldl}, gibt aber eine Liste aller Akkumulatorwerte zurück
		\begin{itemize}
			\item Bspw. \texttt{scanl (*) 1 [1, 3, 5] == [1, 1, 3, 15]}
		\end{itemize}
	\end{itemize}
\end{frame}

\section{Lazy Evaluation}

\begin{frame}{Lazy Evaluation}
	\code{code/ghci-lazy.output}

	\begin{itemize}
		\item Was heißt Lazy Evaluation?
		\item Wieso tritt erst bei der zweiten Eingabe ein Fehler auf?
		\pause
		\item $\leadsto$ Berechnungen finden erst statt, wenn es \emph{absolut} nötig ist
	\end{itemize}
\end{frame}

\begin{frame}{Lazy Evaluation}
      \href{https://wiki.haskell.org/Lazy\_evaluation}{wiki.haskell.org/Lazy\_Evaluation}:

	\begin{displayquote}
		Lazy evaluation means that expressions are not evaluated when they are bound to variables, but their evaluation is \textbf{deferred} until their results are needed by other computations.
	\end{displayquote}

	\begin{itemize}
		\item Auch: \emph{call-by-name} im Gegensatz zu \emph{call-by-value} in bspw. C
		\item Was bringt das?
		\pause
		\item Ermöglicht bspw. arbeiten mit unendlichen Listen
		\item Berechnungen, die nicht gebraucht werden, werden nicht ausgeführt
	\end{itemize}
\end{frame}

\section{Hangman}

\begin{frame}{Hangman}
        Schreibt folgende Funktionen:
	\begin{itemize}
		\item \texttt{showHangman} --- Zeigt aktuellen Spielstand als \texttt{String}
		\begin{itemize}
			\item Definition: \texttt{showHangman word guesses = ...}
			\item Typ: \texttt{showHangman :: String -> [Char] -> String}
		\end{itemize}
		\item \texttt{updateHangman} --- Bildet Usereingabe (als \texttt{String}) und alten Zustand auf neuen Zustand ab
		\begin{itemize}
			\item Definition: \texttt{updateHangman inputLine guesses}
			\item Beispiel: \texttt{updateHangman "{}h"{} "{}aske"{} == "{}haske"{}}
		\end{itemize}
	\end{itemize}
\end{frame}

\begin{frame}{Hangman --- CLI-Framework}
	\code{../demos/CLI.hs}

	\begin{itemize}
		\item \texttt{s} ist der Typ des Spielzustands
		\item Anfänglicher Zustand: \texttt{[]} --- leere Liste an Rateversuchen
                \item \texttt{showHangman :: [Char] -> String}
		\pause
                \item \texttt{updateHangman :: String -> [Char] -> [Char]}
		\pause
                \item \texttt{initHangman :: [Char]}
	\end{itemize}
\end{frame}

\begin{frame}{Hangman --- Beispiele}
	\begin{itemize}
          \item \texttt{showHangman "{}test"{} "{}e"{}} $\Rightarrow$ \texttt{"{}. e . . | e"{}}
          \item \texttt{showHangman "{}test"{} "{}sf"{}} $\Rightarrow$ \texttt{"{}. . s . | s f"{}}
          \item \texttt{updateHangman "{}f"{} "{}abc"{}} $\Rightarrow$ \texttt{"{}fabc"{}}
	\end{itemize}
\end{frame}

\section{Übungsblatt 2}

\begin{frame}{\texttt{where} vs. \texttt{let}}
	\begin{center}
	\texttt{f = let y = 21 in y where y = 50}\\
	Zu was wertet \texttt{f} aus?
	\end{center}

	\pause	

	\begin{figure}
	\begin{tikzpicture}[level distance=0.8cm, sibling distance=1.2cm, every node/.style={scale=0.8}]
		\node (root) {Def}
		child {
			node {\texttt{f x}}
		}
		child {
			node {\texttt{=}}
		}
		child {
			node {Expr}
			child {
				node {\texttt{let}}
			}
			child {
				node {Def}
				child {
					node (def) {\texttt{y}}
				}
				child {
					node {\texttt{=}}
				}
				child {
					node {Expr}
					child { node {\texttt{21}} }
				}
			}
			child {
				node {\texttt{in}}
			}
			child {
				node {Expr}
				child {
					node[draw] (usage) {\texttt{y}}
				}
			}
		}
		child {
			node {\texttt{where}}
		}
		child [sibling distance=2cm] {
			node {Def}
			child [sibling distance=1cm] {
				node {\texttt{y}}
			}
			child [sibling distance=1cm] {
				node {\texttt{=}}
			}
			child [sibling distance=1cm] {
				node {Expr}
				child {
					node {\texttt{50}}
				}
			}
		}
	;
	\draw[->] (usage) edge [bend left=90] (def);
	\end{tikzpicture}
	\end{figure}
\end{frame}

\end{document}
