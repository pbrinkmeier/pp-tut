\documentclass{beamer}
\usetheme{metropolis}

\usetheme{metropolis}

\usepackage[ngerman]{babel}
\usepackage[autostyle=true,german=quotes]{csquotes}
\usepackage[linewidth=1pt]{mdframed}
\usepackage{hyperref}
\usepackage{makecell}
\usepackage{pifont}
\usepackage{tikz}
\usetikzlibrary{positioning, calc, arrows, fit, decorations.pathreplacing, shapes, shapes.multipart, snakes}
\usepackage{verbatim}
\usepackage{textcomp}
\usepackage{centernot}
\usepackage{tabularx}
\usepackage{ulem}
\usepackage{mathpartir}
%\usepackage{pdfpages}

\batchmode

\hypersetup{
	colorlinks,
	urlcolor=blue,
	linkcolor=black % for ToC
}
\newenvironment{qaa}[1]{
	#1

	\begin{mdframed}
		\small
}{
	\end{mdframed}
}

\newcommand{\true}{\ding{51}}
\newcommand{\false}{\ding{55}}
\newcommand{\code}[1]{
	\begin{mdframed}
		\verbatiminput{#1}
	\end{mdframed}
}

% lambda calculus stuff

\newcommand{\app}[2]{#1\;#2}
\newcommand{\lam}[2]{\lambda #1.\, #2}
\newcommand{\appp}[3]{\app{\app{#1}{#2}}{#3}}

\newcommand{\ctrue}{c_\text{true}}
\newcommand{\cfalse}{c_\text{false}}
\newcommand{\cand}{\text{AND}}

\newcommand{\csucc}{\text{succ}}
\newcommand{\cn}[1]{c_{#1}}

\newcommand{\cpair}{\text{pair}}
\newcommand{\cfst}{\text{fst}}
\newcommand{\csnd}{\text{snd}}
\newcommand{\ccurry}{\text{curry}}
\newcommand{\cuncurry}{\text{uncurry}}

\newcommand{\cnil}{\text{nil}}
\newcommand{\ccons}{\text{cons}}
\newcommand{\chead}{\text{head}}
\newcommand{\ctail}{\text{tail}}
\newcommand{\creplicate}{\text{replicate}}

\newcommand{\subst}[3]{(#1)\left[#2\,\to\,#3\right]}


\title{Tutorium 03: Typen und Typklassen}
% \subtitle{}
\author{Paul Brinkmeier}
\institute{Tutorium Programmierparadigmen am KIT}
\date{24. November 2020}

\begin{document}

\begin{frame}
	\titlepage
\end{frame}

\section{Heutiges Programm}
\begin{frame}{Programm}
	\begin{itemize}
                \item Übungsblätter 1 und 2
		\item Wiederholung der Vorlesung: Typen und Typklassen
		\item Datentypen selbst definieren
	\end{itemize}
\end{frame}

\section{Übungsblatt 1}

\begin{frame}{Aufgaben 2 und 3}
  Todo
\end{frame}

\section{Übungsblatt 2}

\begin{frame}
  Todo
\end{frame}

\section{Wiederholung: Typen}

\begin{frame}{Cheatsheet: Listenkombinatoren}
  \begin{itemize}
    \item \texttt{foldr :: (a -> b -> b) -> b -> [a] -> b}
    \item \texttt{foldl :: (b -> a -> b) -> b -> [a] -> b}
    \item \texttt{map :: (a -> b) -> [a] -> [b]}
    \item \texttt{filter :: (a -> Bool) -> [a] -> [a]}
    \item \texttt{zipWith :: (a -> b -> c) -> [a] -> [b] -> [c]}
    \item \texttt{zip :: [a] -> [b] -> [(a, b)]}
    \item \texttt{and, or :: [Bool] -> Bool}
  \end{itemize}

  Idee: Statt Rekursion selbst zu formulieren verwenden wir fertige \enquote{Bausteine}, sogenannte \enquote{Kombinatoren}.
\end{frame}

\begin{frame}{Cheatsheet: Tupel und Konzepte}
  \begin{itemize}
    \item \emph{List comprehension}, \emph{Laziness}
    \item \texttt{[f x | x <- xs, p x]} $\equiv$ \texttt{map f (filter p xs)}\\
      Bspw.: \texttt{[x * x | x <- [1..]]} $\Rightarrow$ \texttt{[1,4,9,16,25,...]}
    \item \emph{Tupel}
    \item \texttt{(,) :: a -> b -> (a, b)} (\enquote{Tupel-Konstruktor})
    \item \texttt{fst :: (a, b) -> a}
    \item \texttt{snd :: (a, b) -> b}
  \end{itemize}
\end{frame}

\begin{frame}{Cheatsheet: Typen}
  \begin{itemize}
    \item \texttt{Char}, \texttt{Int}, \texttt{Integer}, ...
    \item \texttt{String}
    \item \emph{Typvariablen}/\emph{Polymorphe Typen}:
    \begin{itemize}
      \item \texttt{(a, b)}: Tupel
      \item \texttt{[a]}: Listen
      \item \texttt{a -> b}: Funktionen
      \item Vgl. Java: \texttt{List<A>}, \texttt{Function<A, B>}
    \end{itemize}
    \item \emph{Typsynonyme}: \texttt{type String = [Char]}
  \end{itemize}
\end{frame}

\section{Wiederholung: Typklassen}

\section{Typen selbst definieren}

\section{Typklassen definieren und implementieren}

\end{document}
