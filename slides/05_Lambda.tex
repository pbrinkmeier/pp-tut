\documentclass{beamer}

\usetheme{metropolis}

\usepackage[ngerman]{babel}
\usepackage[autostyle=true,german=quotes]{csquotes}
\usepackage[linewidth=1pt]{mdframed}
\usepackage{hyperref}
\usepackage{makecell}
\usepackage{pifont}
\usepackage{tikz}
\usetikzlibrary{positioning, calc, arrows, fit, decorations.pathreplacing, shapes, shapes.multipart, snakes, trees, automata}
\usepackage{verbatim}
\usepackage{textcomp}
\usepackage{centernot}
\usepackage{tabularx}
\usepackage[normalem]{ulem}
\usepackage{mathpartir}
\usepackage{pdfpages}
\usepackage{wasysym}
\usepackage{qtree}
\usepackage{pgfplots}

\batchmode

\hypersetup{
	colorlinks,
	urlcolor=blue,
	linkcolor=black % for ToC
}
\newenvironment{qaa}[1]{
	#1

	\begin{mdframed}
		\small
}{
	\end{mdframed}
}

\newcommand{\true}{\ding{51}}
\newcommand{\false}{\ding{55}}
\newcommand{\code}[1]{
	\begin{mdframed}
		\verbatiminput{#1}
	\end{mdframed}
}

% lambda calculus stuff

\newcommand{\app}[2]{#1\;#2}
\newcommand{\appr}[2]{\app{\uline{#1}}{\uwave{#2}}}
\newcommand{\lam}[2]{\lambda #1.\, #2}
\newcommand{\appp}[3]{\app{\app{#1}{#2}}{#3}}
\newcommand{\apppp}[4]{\app{\app{\app{#1}{#2}}{#3}}{#4}}
\newcommand{\apppr}[3]{\app{\appr{#1}{#2}}{#3}}

\newcommand{\lamlet}[3]{\texttt{let} \;\; #1 = #2 \;\; \texttt{in} \;\; #3}

\newcommand{\ctrue}{c_\text{true}}
\newcommand{\cfalse}{c_\text{false}}
\newcommand{\cand}{\text{AND}}

\newcommand{\csucc}{\text{succ}}
\newcommand{\cn}[1]{c_{#1}}

\newcommand{\cpair}{\text{pair}}
\newcommand{\cfst}{\text{fst}}
\newcommand{\csnd}{\text{snd}}
\newcommand{\ccurry}{\text{curry}}
\newcommand{\cuncurry}{\text{uncurry}}

\newcommand{\cnil}{\text{nil}}
\newcommand{\ccons}{\text{cons}}
\newcommand{\chead}{\text{head}}
\newcommand{\ctail}{\text{tail}}
\newcommand{\creplicate}{\text{replicate}}

\newcommand{\subst}[3]{(#1)\left[#2\,\to\,#3\right]}

% unification etc.

\newcommand{\functor}[2]{\texttt{#1}(#2)}
\newcommand{\pa}[1]{\texttt{#1}}
\newcommand{\plsunify}{\stackrel{!}{=}}
\newcommand{\unifier}[1]{\left[#1\right]}
\newcommand{\su}[2]{#1 \; \text{\pointer} \; #2}


\title{Tutorium 05: $\lambda$-Kalkül}
% \subtitle{}
\author{Paul Brinkmeier}
\institute{Tutorium Programmierparadigmen am KIT}
\date{29. November 2022}

\begin{document}

\begin{frame}
	\titlepage
\end{frame}

\section{Übungsblätter}

\section{Wiederholung}

\begin{frame}{Cheatsheet: Lambda-Kalkül/Basics}
  \begin{itemize}
    \item Terme $t$: Variable ($x$), Funktion ($\lambda x . t$), Anwendung ($t \; t$)
    \item \emph{$\alpha$-Äquivalenz}: Gleiche Struktur
    \item \emph{$\eta$-Äquivalenz}: Unterversorgung
    \item \emph{Freie Variablen}, \emph{Substitution}, \emph{RedEx}
    \item \emph{$\beta$-Reduktion}: \\
          $(\lambda{}p.b)$ $t \Rightarrow b\left[p\rightarrow{}t\right]$
  \end{itemize}
\end{frame}

\begin{frame}{$\lambda$-Terme sind Bäume}
    \begin{columns}
        \begin{column}{0.3\textwidth}
            \begin{figure}
                \begin{tikzpicture}[node font=\ttfamily, level distance=0.8cm, sibling distance=1.2cm, every node/.style={scale=0.8}]
                    \node (root) {$\lambda$f}
                    child {
                        node {$\lambda$x}
                        child {
                            node {@}
                            child {
                                node {f}
                            }
                            child {
                                node {@}
                                child {
                                    node {f}
                                }
                                child {
                                    node {x}
                                }
                            }
                        }
                    };
                \end{tikzpicture}
            \end{figure}
        \end{column}
        \begin{column}{0.3\textwidth}
            \begin{figure}
                \begin{tikzpicture}[node font=\ttfamily, level distance=0.8cm, sibling distance=1.2cm, every node/.style={scale=0.8}]
                    \node (root) {$\lambda$f}
                    child {
                        node {$\lambda$x}
                        child {
                            node {@}
                            child {
                                node {f}
                            }
                            child {
                                node {@}
                                child {
                                    node {f}
                                }
                                child {
                                    node {@}
                                    child {
                                        node {f}
                                    }
                                    child {
                                        node {@}
                                        child {
                                            node {f}
                                        }
                                        child {
                                            node {x}
                                        }
                                    }
                                }
                            }
                        }
                    };
                \end{tikzpicture}
            \end{figure}
        \end{column}
        \begin{column}{0.3\textwidth}
            \begin{figure}
                \begin{tikzpicture}[node font=\ttfamily, level distance=0.8cm, sibling distance=1.2cm, every node/.style={scale=0.8}, level 5/.style={sibling distance=0.6cm}]
                    \node (root) {$\lambda$x}
                    child {
                        node {$\lambda$y}
                        child {
                            node {$\lambda$z}
                            child {
                                node {@}
                                child {
                                    node {@}
                                    child { node {x} }
                                    child { node {z} }
                                }
                                child {
                                    node {@}
                                    child { node {y} }
                                    child { node {z} }
                                }
                            }
                        }
                    };
                \end{tikzpicture}
            \end{figure}
        \end{column}
    \end{columns}

    \vfill

    \begin{columns}
        \begin{column}{0.3\textwidth}
            \pause
            \center
            $\lam{f}{\lam{x}{\app{f}{(\app{f}{x})}}}$
        \end{column}
        \begin{column}{0.3\textwidth}
            \pause
            \center
            $\lam{f}{\lam{x}{\app{f}{(\app{f}{(\app{f}{(\app{f}{x})}})})}}$
        \end{column}
        \begin{column}{0.3\textwidth}
            \pause
            \center
            $\lam{x}{\lam{y}{\lam{z}{\app{\app{x}{z}}{(\app{y}{z})}}}}$
        \end{column}
    \end{columns}
\end{frame}

\begin{frame}{Klammerung von Lambda-Termen}
  \begin{columns}
    \column{0.5\textwidth}
    \begin{figure}
      \begin{tikzpicture}[node font=\ttfamily, level distance=0.8cm, sibling distance=1.2cm, every node/.style={scale=0.8}, level 5/.style={sibling distance=0.6cm}]
          \node (root) {@}
            child { node {@}
              child { node {f} }
              child { node {x} }
            }
            child { node {y} }
          ;
      \end{tikzpicture}
    \end{figure}

    \column{0.5\textwidth}
    \begin{figure}
      \begin{tikzpicture}[node font=\ttfamily, level distance=0.8cm, sibling distance=1.2cm, every node/.style={scale=0.8}, level 5/.style={sibling distance=0.6cm}]
          \node (root) {@}
            child { node {f} }
            child { node {@}
              child { node {x} }
              child { node {y} }
            }
          ;
      \end{tikzpicture}
    \end{figure}
  \end{columns}
  \begin{columns}
    \column{0.5\textwidth}
    \pause
    \center
    $$
    \app{(\app{f}{x})}{y} = \appp{f}{x}{y}
    $$
    \small
    \enquote{$f$ angewendet auf $x$ und $y$}

    \column{0.5\textwidth}
    \pause
    \center
    $$
    \app{f}{(\app{x}{y})} \neq \appp{f}{x}{y}
    $$
    \small
    \enquote{$f$ angewendet auf das Ergebnis von $\app{x}{y}$}
  \end{columns}
  \vfill
  \pause
  \begin{itemize}
    \item \emph{Funktionsanwendung} ist linksassoziativ (linksgeklammert)
    \item $\leadsto$ Wie bei Haskell!
  \end{itemize}
\end{frame}

\begin{frame}{Klammerungsbeispiel}
  \code{code/sum.hs}
  \small
  \begin{align*}
    sum = \lam{a}{\lam{b}{\lam{f}{\apppp{ifZero}{&(\appp{sub}{b}{(\app{succ}{a})})\\}{&\cn{0}\\}{&(\appp{add}{(\app{f}{a})}}{(\apppp{sum}{(\app{succ}{a})}{b}{f})})}}}
  \end{align*}
  \begin{columns}
    \column{0.5\textwidth}
    \begin{figure}
      \begin{tikzpicture}[node font=\ttfamily, level distance=0.5cm, sibling distance=1.2cm, every node/.style={scale=0.8}]
        \node (root) {@}
          child { node {@}
            child { node {@}
              child { node {sum} }
              child { node {@}
                child { node {succ} }
                child { node {a} }
              }
            }
            child { node {b} }
          }
          child { node {f} };
      \end{tikzpicture}
    \end{figure}

    \column{0.5\textwidth}
    \begin{figure}
      \begin{tikzpicture}[node font=\ttfamily, level distance=0.5cm, sibling distance=2cm, every node/.style={scale=0.8}, level 1/.style={sibling distance=2.4cm}, level 2/.style={sibling distance=1.2cm}]
        \node (root) {@}
          child { node {@}
            child { node {sub} }
            child { node {b} }
          }
          child { node {@}
            child { node {succ} }
            child { node {a} }
          };
      \end{tikzpicture}
    \end{figure}
  \end{columns}
\end{frame}

\begin{frame}{Klammerung von Funktionstypen}
  \code{code/sum.hs}

  \center
  \texttt{sum :: Int -> Int -> (Int -> Int) -> Int}\\
  \texttt{sum :: Int -> (Int -> ((Int -> Int) -> Int))}

  \begin{columns}
    \column{0.5\textwidth}
    \begin{figure}
      \begin{tikzpicture}[node font=\ttfamily, level distance=0.5cm, sibling distance=1.2cm, every node/.style={scale=0.8}]
        \node (root) {@}
          child { node {@}
            child { node {@}
              child { node {sum} }
              child { node {a+1} }
            }
            child { node {b} }
          }
          child { node {f} };
      \end{tikzpicture}
    \end{figure}

    \column{0.5\textwidth}
    \begin{figure}
      \begin{tikzpicture}[node font=\ttfamily, level distance=0.5cm, sibling distance=1.2cm, every node/.style={scale=0.8}]
        \node (root) {->}
          child { node {Int} }
          child { node {->}
            child { node {Int} }
            child { node {->}
              child {node {->}
                child { node {Int} }
                child { node {Int} }
              }
              child {node {Int} }
            }
          };
      \end{tikzpicture}
    \end{figure}
  \end{columns}

  \pause
  \begin{itemize}
    \item Auch Typterme sind Bäume
    \item \emph{Funktionstypen} sind rechtsassoziativ
  \end{itemize}
\end{frame}

\section{Church-Zahlen}

\begin{frame}{Peano-Axiome}
	\begin{eqnarray*}
		c_0 &= c_0\\
		c_1 &= s (c_0)\\
		c_2 &= s (s (c_0))\\
		c_3 &= s (s (s (c_0)))\\
		c_8 &= s (s (s (s (s (s (s (s (c_0))))))))
	\end{eqnarray*}

	\begin{enumerate}
		\item Die 0 ist Teil der natürlichen Zahlen
		\item Wenn $n$ Teil der natürlichen Zahlen ist,\\
	 	      ist auch $s(n) = n + 1$ Teil der natürlichen Zahlen
	\end{enumerate}
\end{frame}

\begin{frame}{Church-Zahlen}
	\begin{itemize}
		\item \enquote{Zahlen} im $\lambda$-Kalkül werden durch Funktionen in Normalform dargestellt
		\item $c_n$ $f$ $x =$ $f$ $n$-mal angewendet auf $x$
		\item Bspw. $(c_3$ $g$ $y) = g$ $(g $ $(g$ $y)) = g^3$ $y$\\
		      Mit $c_3 = \lambda{}f.\lambda{}x.f$ $(f $ $(f$ $x))$
		\item Schreibt eine $\lambda$-Funktion $succ$, die eine Church-Zahl nimmt und zu deren Nachfolger auswertet
              \end{itemize}
              \pause

              \begin{equation*}
                \csucc = \lam{n}{\lam{s}{\lam{z}{
                  \app{s}{(
                    \app{n}{\app{s}{z}}
                  )}
                }}}
              \end{equation*}
\end{frame}

\section{Auswertungsstrategien}

\begin{frame}{Auswertungsstrategien}
        \begin{equation*}
          \overbrace{\app{
              \uline{\csucc}
            }{
              \overbrace{(\app{\uline{\csucc}}{\cn{0}})}^{\text{Redex 2}}
            }}^{\text{Redex 1}}
        \end{equation*}

        mit

        \begin{eqnarray*}
          \cn{0} =& \lam{s}{\lam{z}{z}} \\
          \csucc =& \lam{n}{\lam{s}{\lam{z}{\app{s}{(\app{\app{n}{s}}{z})}}}}
        \end{eqnarray*}

	\begin{itemize}
		\item Welcher Redex soll zuerst ausgewertet werden?
		\item $\leadsto$ verschiedene Auswertungsstrategien
	\end{itemize}
\end{frame}

\begin{frame}{Normalreihenfolge}
        \begin{eqnarray*}
          &
          \app{
            \uline{\csucc}
          }{
            (\app{\uline{\csucc}}{\cn{0}})
          } \\
          \Rightarrow_\beta&
          \lam{s}{\lam{z}{\app{s}{(\app{\app{(\app{\uline{\csucc}}{\cn{0}})}{s}}{z})}}} \\
          \Rightarrow_\beta&
          \lam{s}{\lam{z}{
            \app{s}{
              (\app{\app{\uline{(\lam{s}{\lam{z}{
                \app{s}{(\app{\app{\uline{\cn{0}}}{s}}{z})}
              }})}}{s}}{z})
            }
          }} \\
          \Rightarrow_\beta^2&
          \lam{s}{\lam{z}{
            \app{s}{(\app{s}{(\app{\app{\uline{\cn{0}}}{s}}{z})})}
          }} \\
          \Rightarrow_\beta^2&
          \lam{s}{\lam{z}{
            \app{s}{(\app{s}{z})}
          }} \centernot\implies
        \end{eqnarray*}

        \vfill

        Normalreihenfolge: Linkester Redex zuerst.
\end{frame}

\begin{frame}{Call-by-Name}
        \begin{eqnarray*}
          &
          \app{
            \uline{\csucc}
          }{
            (\app{\uline{\csucc}}{\cn{0}})
          } \\
          \Rightarrow_\beta&
          \lam{s}{\lam{z}{\app{s}{(\app{\app{(\app{\uline{\csucc}}{\cn{0}})}{s}}{z})}}} \centernot\implies_{\text{CbN}}
        \end{eqnarray*}

        \vfill

        Call-by-Name: Linkester Redex zuerst, aber:

        \begin{itemize}
          \item Funktionsinhalte werden nicht weiter reduziert
          \item $\leadsto$ Betrachte nur Redexe, die nicht von einem $\lambda$ umgeben sind
          \item So funktioniert auch Laziness in Haskell (grob)
        \end{itemize}
\end{frame}

\begin{frame}{Call-by-Value}
        \begin{eqnarray*}
          &
          \app{
            \uline{\csucc}
          }{
            (\app{\uline{\csucc}}{\cn{0}})
          } \\
          \Rightarrow_\beta&
          \app{
            \uline{\csucc}
          }{
            (\lam{s}{\lam{z}{
              \app{s}{
                (\app{\app{\uline{\cn{0}}}{s}}{z})
              }
            }})
          } \\
          \Rightarrow_\beta&
          \lam{s}{\lam{z}{
          \app{\app{
            \uline{(\lam{s}{\lam{z}{
              \app{s}{
                (\app{\app{\uline{\cn{0}}}{s}}{z})
              }
            }})}
          }{s}}{z}
          }} \centernot\implies_\text{CbV}
        \end{eqnarray*}
        \vfill

        Call-by-Value: Linkester Redex zuerst, aber:

        \begin{itemize}
          \item Funktionsinhalte werden nicht weiter reduziert
          \item $\leadsto$ Betrachte nur Redexe, die nicht von einem $\lambda$ umgeben sind
          \item Berechne Argumente vor dem Einsetzen
          \item $\leadsto$ Betrachte nur Redexe, deren Argument \emph{unter CbV} nicht weiter reduziert werden muss
        \end{itemize}
\end{frame}

\begin{frame}{Cheatsheet: Lambda-Kalkül/Fortgeschritten}
  \begin{itemize}
    \item Auswertungsstrategien (von lässig nach streng):
    \begin{itemize}
      \item \emph{Volle $\beta$-Reduktion}
      \item \emph{Normalreihenfolge}
      \item \emph{Call-by-Name}
      \item \emph{Call-by-Value}
    \end{itemize}
    \item Datenstrukturen:
    \begin{itemize}
      \item \emph{Church-Booleans}
      \item \emph{Church-Zahlen}
      \item \emph{Church-Listen}
    \end{itemize}
    \item Rekursion durch \emph{Y-Kombinator}
  \end{itemize}
\end{frame}

\section{Klausuraufgaben zum $\lambda$-Kalkül}

\begin{frame}{19SS A4 --- SKI-Kalkül (13P.)}
	\begin{eqnarray*}
          S =& \lam{x}{\lam{y}{\lam{z}{\appp{x}{z}{(\app{y}{z})}}}} \\
		K =& \lam{x}{\lam{y}{x}} \\
		I =& \lam{x}{x}
	\end{eqnarray*}

	\begin{itemize}
		\item SKI-Kalkül kann alles, was $\lambda$-Kalkül auch kann, allein mit den Kombinatoren $S$, $K$ und $I$
		\item Definiere $U = \lam{x}{\appp{x}{S}{K}}$
		\item Aufgabe: Beweise, dass man $S$, $K$ und $I$ durch $U$ darstellen kann:
		\pause
		\begin{itemize}
			\item $\appp{U}{U}{x} \stackrel{?}{\implies} x$
                        \item $\app{U}{(\app{U}{(\app{U}{U})})} = \app{U}{(\app{U}{I})} \stackrel{?}{\implies} K$
                        \item $\app{U}{(\app{U}{(\app{U}{(\app{U}{U})})})} = \app{U}{K} \stackrel{?}{\implies} S$
		\end{itemize}
	\end{itemize}
\end{frame}

\begin{frame}{18SS A4 --- Currying im $\lambda$-Kalkül (13P.)}
	\begin{eqnarray*}
		\cpair =& \lam{a}{\lam{b}{\lam{f}{\appp{f}{a}{b}}}} \\
                \cfst  =& \lam{p}{\app{p}{(\lam{x}{\lam{y}{x}})}} \\
                \csnd  =& \lam{p}{\app{p}{(\lam{x}{\lam{y}{y}})}} \\
                \app{\cfst}{(\appp{\cpair}{a}{b})} =& a \\
                \app{\csnd}{(\appp{\cpair}{a}{b})} =& b \\
	\end{eqnarray*}

	\begin{itemize}
		\item Schreibe $\ccurry$ und $\cuncurry$, sodass:
		\begin{itemize}
                  \item $\appp{(\app{\ccurry}{f})}{a}{b} = \appp{f}{(\appp{\cpair}{a}{b})}$
                  \item $\app{(\app{\cuncurry}{g})}{(\appp{\cpair}{a}{b})} = \appp{g}{a}{b}$
		\end{itemize}
	\end{itemize}
\end{frame}

\begin{frame}{18WS A5 --- Listen im $\lambda$-Kalkül (10P.)}
	\begin{eqnarray*}
		\cnil  =& \lam{n}{\lam{c}{n}} \\
                \ccons =& \lam{x}{\lam{xs}{\lam{n}{\lam{c}{(\appp{c}{x}{xs})}}}}
	\end{eqnarray*}

	\begin{itemize}
		\item Schreibe $head$ und $tail$, sodass:
		\begin{itemize}
                  \item $\app{\chead}{(\appp{\ccons}{A}{B})} \stackrel{*}{\implies} A$
                  \item $\app{\ctail}{(\appp{\ccons}{A}{B})} \stackrel{*}{\implies} B$
		\end{itemize}
		\pause
		\item Schreibe $\creplicate$, sodass:
		\begin{itemize}
                  \item $\appp{\creplicate}{\cn{n}}{A} = \underbrace{\appp{\ccons}{A}{(\appp{\ccons}{A}{... (\appp{\ccons}{A}{\cnil})})}}_\text{$n$ mal}$
		\end{itemize}
              \item Erinnerung: $\appp{\cn{n}}{f}{x} = \underbrace{\app{f}{(\app{f}{... (\app{f}{x})})}}_\text{$n$ mal}$
		\pause
              \item Werte aus: $\appp{\creplicate}{\cn{3}}{A} \stackrel{*}{\implies} ?$
	\end{itemize}
\end{frame}

\end{document}
