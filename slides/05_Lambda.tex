\documentclass{beamer}
\usetheme{metropolis}

\usepackage[ngerman]{babel}
\usepackage[autostyle=true,german=quotes]{csquotes}
\usepackage[linewidth=1pt]{mdframed}
\usepackage{hyperref}
\usepackage{makecell}
\usepackage{pifont}
\usepackage{tikz}
\usetikzlibrary{positioning, calc, arrows, fit, decorations.pathreplacing, shapes, shapes.multipart, snakes}
\usepackage{verbatim}
\usepackage{tabularx}
\usepackage{textcomp}
\usepackage{centernot}
\usepackage{amsmath}
%\usepackage{pdfpages}

\batchmode

\hypersetup{
	colorlinks,
	urlcolor=blue,
	linkcolor=black % for ToC
}
\newenvironment{qaa}[1]{
	#1

	\begin{mdframed}
		\small
}{
	\end{mdframed}
}

\newcommand{\true}{\ding{51}}
\newcommand{\false}{\ding{55}}
\newcommand{\code}[1]{
	\begin{mdframed}
		\verbatiminput{#1}
	\end{mdframed}
}

\title{Tutorium 05: $\lambda$-Kalkül}
% \subtitle{}
\author{Paul Brinkmeier}
\institute{Tutorium Programmierparadigmen am KIT}
\date{18. November 2019}

\begin{document}

\begin{frame}
	\titlepage
\end{frame}

\section{Heutiges Programm}
\begin{frame}{Programm}
	\begin{itemize}
		\item Übungsblatt 4
		\item $\lambda$-Kalkül: Basics + Church-Zahlen
		\pause
		\item $\lambda$-Kalkül in Haskell
	\end{itemize}
\end{frame}

\section{Übungsblatt 4}

\begin{frame}{2.1, 2.3 --- AST: Datenstruktur}
	\code{demos/AstType.hs}

	\begin{itemize}
		\item \texttt{t} ist Typvariable, um bspw. \texttt{Int}s als Namen zuzulassen
		\item Das kommt bspw. bei Compiler-Optimierungen zum Einsatz
	\end{itemize}
\end{frame}

\begin{frame}{2.2 --- AST: Auswertung}
	\code{demos/AstEval.hs}
\end{frame}

\begin{frame}{2.3 --- AST: Boolsche Ausdrücke}
	\code{demos/AstEval2.hs}

	\begin{itemize}
		\item Aufgabe sorgfältig lesen, nur 0 ist \enquote{falsey} in C
		\item $\leadsto$ kann einem in der Klausur in den Arsch beißen
	\end{itemize}
\end{frame}

\begin{frame}{2.4 --- AST: \texttt{Show}}
	\code{demos/AstShow.hs}

	\begin{itemize}
		\item \texttt{Show t => Show (Exp t)} $\Leftrightarrow$ \enquote{Wenn man \texttt{t}s anzeigen kann, kann man auch \texttt{Exp t}s anzeigen}
	\end{itemize}
\end{frame}

\section{Wiederholung}

\begin{frame}{Algebraische Datentypen}
	\code{demos/DataExamples.hs}

	\begin{itemize}
		\item Keyword \text{data} definiert \emph{neuen} Typ
		\item \enquote{\texttt{enum} auf Meth}
	\end{itemize}
\end{frame}

\begin{frame}{Typklassen}
	\code{demos/TypeClassExamples.hs}

	\begin{itemize}
		\item Typklassen stellen globale Operationen für Typen bereit
		\item Bspw. \texttt{Eq} und \texttt{Ord} für Vergleiche, \texttt{Enum} für Aufzählbarkeit
	\end{itemize}
\end{frame}

\section{$\lambda$-Kalkül}

\begin{frame}{$\lambda$-Kalkül}
	\begin{itemize}
		\item Funktionales Gegenstück zur Turingmaschine
		\item Entsprechend theoretisch
		\item Wurde u.a. genutzt um Unlösbarkeit des Halteproblems zu zeigen
		\item Gibt saftig Punkte in der Klausur
		\begin{itemize}
			\item 13P. im 19SS
			\item 10P. (+15P.) im 18WS
			\item 20P. (+15P.) im 18SS
		\end{itemize}
		\item Nicht kompliziert aber \enquote{schwierig} (wie bspw. Go oder Schach)
	\end{itemize}
\end{frame}

\begin{frame}{$\lambda$-Terme}
	Ein Term im $\lambda$-Kalkül hat eine der drei folgenden Formen:

	\vspace{0.5cm}

	\begin{tabularx}{\textwidth}{ X | X | X }
		\textbf{Notation} & \textbf{Besteht aus}                      & \textbf{Bezeichnung} \\
		\hline
		$x$               & $x$ : Variablenname                       & Variable             \\
		\hline
		$\lambda{}p.b$    &
			\begin{tabular}[t]{@{}c@{}}$p$ : Variablenname\\$b$ : $\lambda$-Term\end{tabular}
									      & Abstraktion          \\
		\hline
		$f$ $a$           & $f$, $a$ : $\lambda$-Terme                & Funktionsanwendung   \\
	\end{tabularx}

	\vspace{0.5cm}

	\begin{itemize}
		\item \enquote{$\lambda$-Term}: rekursive Datenstruktur
		\item Semantik definieren wir später
		\pause
		\item Jetzt: Ergänzt das Modul \texttt{Lambda} um die fehlenden Typen
		\begin{itemize}
			\item +Fragen zur ÜB-Korrektur
		\end{itemize}
	\end{itemize}
\end{frame}

\begin{frame}{$\lambda$-Terme in Haskell}
	\code{demos/Lambda.hs}

	\begin{itemize}
		\item \url{//github.com/pbrinkmeier/pp-tut}
		\item Modul \texttt{x} liegt in \texttt{slides/demos/x.hs}
	\end{itemize}
\end{frame}

\newcommand{\aeq}{\stackrel{\alpha}{=}}
\newcommand{\naeq}{\stackrel{\alpha}{\neq}}
\newcommand{\eeq}{\stackrel{\eta}{=}}

\newcolumntype{s}{>{\hsize=.8\hsize}X}

\begin{frame}{Begriffe im $\lambda$-Kalkül}
	\fontsize{9pt}{13}\selectfont

	\begin{tabularx}{\textwidth}{ s | X | X }
		\textbf{Begriff} & \textbf{Formel} & \textbf{Bedeutung} \\
		\hline
		$\alpha$-Äquivalenz & $t_1 \aeq t_2$ & $t_1$, $t_2$ sind gleicher Struktur \\
		\hline
		$\eta$-Äquivalenz & $\lambda{}x.f$ $x \eeq f$ & \enquote{Unterversorgung} \\
		\hline
		Freie Variablen & $fv(\lambda{}p.b) = { b }$ & Menge der nicht durch $\lambda$s gebundenen Variablen \\
		\hline
		Substitution & $(\lambda{}p.b)\left[b\rightarrow{}c\right]=\lambda{}p.c$ & Ersetzung nicht-freier Variablen \\
		\hline
		Redex & $(\lambda{}p.b)$ $t$ & \enquote{Reducible expression} \\
		\hline
		$\beta$-Reduktion & $(\lambda{}p.b)$ $t \Rightarrow b\left[p\rightarrow{}t\right]$ & \enquote{Funktionsanwendung} \\
	\end{tabularx}
\end{frame}

\subsection{Substitution, Freie Variablen}

\begin{frame}{Freie Variablen}
	\begin{itemize}
		\item $fv(t)$ bezeichnet die frei vorkommenden Variablen im Term $t$
		\item Frei vorkommend $\approx$ nicht durch ein $\lambda$ gebunden
		\begin{itemize}
			\item $fv(x) = {x}$, wenn $x$ Variable
			\item $fv(f$ $x) = fv(f) \cup fv(x)$
			\item $fv(\lambda{}p.b) = fv(b) \setminus p$
			\item $fv(\lambda{}x.x) = \emptyset$
			\item $fv(\lambda{}x.y) = {y}$
		\end{itemize}
		\pause
		\item Implementiert \texttt{fv :: LambdaTerm -> Set String}
		\begin{itemize}
			\item Benutzt \texttt{Set}, \texttt{union}, \texttt{delete} und \texttt{fromList} aus \texttt{Data.Set}
		\end{itemize}
	\end{itemize}
\end{frame}

\begin{frame}{Substitution}
	\begin{itemize}
		\item Substitution ersetzt alle \emph{freien} Variablen in einem Term
		\item $t\left[a \to b\right]$ --- Ersetze $a$ durch $b$ in $t$
		\item Beispiele:
		\begin{itemize}
			\item $a\left[a \to b\right] = b$
			\item $a\left[b \to c\right] = a$
			\item $(f$ $x)\left[f \to g\right]\left[x \to y\right] = g$ $y$
			\item $(\lambda{}x.f$ $x)\left[x \to y\right] = \lambda{}x.f$ $x$ ($x$ ist nicht frei)
			\item $(\lambda{}x.f$ $x)\left[f \to g\right] = \lambda{}x.g$ $x$ ($f$ ist frei)
		\end{itemize}
		\pause
		\item Implementiert\\
		      \texttt{substitute :: (String, Term) -> Term -> Term}
		\begin{itemize}
			\item \texttt{type Term = LambdaTerm}
			\item \texttt{fv} braucht ihr dafür nicht
		\end{itemize}
	\end{itemize}
\end{frame}

\subsection{Äquivalenz}

\begin{frame}{$\alpha$-Äquivalenz}
	\begin{itemize}
		\item $t_1 \aeq t_2$ --- Strukturelle Äquivalenz der Terme $t_1$ und $t_2$
		\item Umformung von $t_1$ in $t_2$ allein durch (konsistente!) Substitution der Variablen möglich
		\begin{itemize}
			\item Es gilt für Variablen $a$ und $b$: $a \aeq a\left[a \to b\right] = b$
		\end{itemize}
		\pause
		\item Bspw.:
		\begin{itemize}
			\item $x \aeq y$, denn $x \aeq x\left[x \to y\right] = y$
			\item $\lambda{}x.x \aeq \lambda{}y.y$
			\item $f$ $(\lambda{}x.y) \aeq f$ $(\lambda{}p.y)$
			\item $\lambda{}x.y \naeq \lambda{}x.z$
		\end{itemize}
		\pause
		\item Aufgabe: Implementiert \texttt{instance Eq Term} als $\alpha$-Äquivalenz
		\begin{itemize}
			\item Benutzt \texttt{substitute}!
		\end{itemize}
	\end{itemize}
\end{frame}

\begin{frame}{$\eta$-Äquivalenz}
	\begin{itemize}
		\item $\lambda{}x.f$ $x \eeq f$, wenn $x \notin fv(f)$
		\item Wie bei Haskell:\\
			\texttt{all list = foldl (\&\&) True list} $\Leftrightarrow$\\
			\texttt{all = \textbackslash{}list -> foldl (\&\&) True list} $\Leftrightarrow$\\
			\texttt{all = foldl (\&\&) True}
		\item Also:
		\begin{itemize}
			\item $\eta$-Äquivalenz: eher Umformungsschritt als Gleichheitskriterium
			\item Formelle Definition von Unterversorgung
		\end{itemize}
	\end{itemize}
\end{frame}

\subsection{Redex, $\beta$-Reduktion}

\begin{frame}{$\beta$-Reduktion}
	\begin{itemize}
		\item Bisher: $\lambda$-Terme als (seltsame) Datenstruktur\\
		      Jetzt: Ausführungssemantik
		\pause
		\item RedEx: \enquote{Reducible expression} $\Leftrightarrow$\\
		      Funktionsanwendung ($f$ $a$), mit $f = \lambda{}p.b$
	      \item $(\lambda{}p.b)$ $a\pause \implies b\left[p \to a\right]$
		\pause
		\item \enquote{Ausführung} (besser: Auswertung) von $\lambda$-Termen: Anwenden der $\beta$-Reduktion, bis Term \enquote{konvergiert}
		\item Term konvergiert $\approx$ Normalform $\approx$ enthält keinen Redex mehr
		\begin{itemize}
			\item Notation: $t \centernot\implies$
		\end{itemize}
		\pause
		\item $id$ $a = (\lambda{}x.x)$ $a \implies x\left[x \to a\right] = a \centernot\implies$
	\end{itemize}
\end{frame}

\begin{frame}{Auswertungsstrategien}
	\begin{itemize}
		\item Welcher Redex soll zuerst ausgewertet werden?
		\item $\leadsto$ verschiedene Auswertungsstrategien
	\end{itemize}
	\pause
	\vspace{1cm}
	\begin{itemize}
		\item \textbf{Volle $\beta$-Reduktion} --- Beliebiger Redex
		\item \textbf{Normalreihenfolge} --- \enquote{Linkester} Redex
		\item \textbf{Call-by-Name} --- \emph{Nur} \enquote{Linkester Redex}
		\item \textbf{Call-by-Value} --- \enquote{Linkester Redex}, der eine Normalform als Argument hat
	\end{itemize}
\end{frame}

\begin{frame}{Normalreihenfolge}
	\code{demos/LambdaN.hs}
	\begin{itemize}
		\item Implementiert\\
		      \texttt{normalBeta :: LambdaTerm -> LambdaTerm}
		\item Führt einen $\beta$-Reduktionsschritt in Normalreihenfolge aus
		\item Wenn kein Redex vorkommt, wird derselbe Term zurückgegeben
		\item Bindet \texttt{LambdaShow} ein für \texttt{instance Show LambdaTerm}
	\end{itemize}
\end{frame}

\section{Church-Zahlen im $\lambda$-Kalkül}

\begin{frame}{Peano-Axiome}
	\begin{eqnarray*}
		c_0 &= ?\\
		c_1 &= s (c_0)\\
		c_2 &= s (s (c_0))\\
		c_3 &= s (s (s (c_0)))\\
		c_8 &= s (s (s (s (s (s (s (s (c_0))))))))
	\end{eqnarray*}

	\begin{enumerate}
		\item Die 0 ist Teil der natürlichen Zahlen
		\item Wenn $n$ Teil der natürlichen Zahlen ist,\\
	 	      ist auch $s(n) = n + 1$ Teil der natürlichen Zahlen
	\end{enumerate}
\end{frame}

\begin{frame}{Church-Zahlen}
	\begin{itemize}
		\item \enquote{Zahlen} im $\lambda$-Kalkül werden durch Funktionen in Normalform dargestellt
		\item $n$ $f$ $x =$ $f$ $n$-mal angewendet auf $x$
		\item Bspw. $(3$ $g$ $y) = g$ $(g $ $(g$ $y)) = g^3$ $y$\\
		      Mit $3 = \lambda{}f.\lambda{}x.f$ $(f $ $(f$ $x))$
		\item Schreibt eine $\lambda$-Funktion $succ$, die eine Church-Zahl nimmt und zu deren Nachfolger auswertet
		\pause
		\item Übertragt die Funktion in euren Haskell-Code und wertet $succ$ $c_0$ durch wiederholtes Anwenden von \texttt{normalBeta} aus
		\item Vergleicht euer Ergebnis mit dem von Wavelength
		\begin{itemize}
			\item \url{//pp.ipd.kit.edu/lehre/misc/lambda-ide/Wavelength.html}
		\end{itemize}
	\end{itemize}
\end{frame}

\end{document}
