\documentclass{beamer}
\usetheme{metropolis}

\usepackage[ngerman]{babel}
\usepackage[autostyle=true,german=quotes]{csquotes}
\usepackage[linewidth=1pt]{mdframed}
\usepackage{hyperref}
\usepackage{makecell}
\usepackage{pifont}
\usepackage{tikz}
\usetikzlibrary{positioning, calc, arrows, fit, decorations.pathreplacing, shapes, shapes.multipart, snakes}
\usepackage{verbatim}
\usepackage{tabularx}
\usepackage{textcomp}
\usepackage{centernot}
\usepackage{amsmath}
\usepackage{xcolor}
\usepackage{tikz}
%\usepackage{pdfpages}

\batchmode

\hypersetup{
	colorlinks,
	urlcolor=blue,
	linkcolor=black % for ToC
}
\newenvironment{qaa}[1]{
	#1

	\begin{mdframed}
		\small
}{
	\end{mdframed}
}

\newcommand{\true}{\ding{51}}
\newcommand{\false}{\ding{55}}
\newcommand{\code}[1]{
	\begin{mdframed}
		\verbatiminput{#1}
	\end{mdframed}
}

\title{Tutorium 08: Prolog}
% \subtitle{}
\author{Paul Brinkmeier}
\institute{Tutorium Programmierparadigmen am KIT}
\date{09. Dezember 2019}

\begin{document}

\begin{frame}
	\titlepage
\end{frame}

\section{Heutiges Programm}

\begin{frame}{Programm}
	\begin{itemize}
		\item Prolog-Grundlagen
		\item Aufgaben zu Prolog
	\end{itemize}
\end{frame}

\section{Prolog-Grundlagen}

\section{Aufgaben}

\subsection{Mathematiker-WG}

\begin{frame}{Mathematiker-WG}
	\begin{figure}
		\includegraphics[width=0.8\textwidth]{images/mathematiker-wg}
	\end{figure}
\end{frame}

\begin{frame}{Mathematiker-WG}
	\begin{itemize}
		\item Alice, Bob und Carl ziehen in eine WG
		\item Die drei sind Mathematiker;\\jeder will eine eigene Zahl von 1 bis 7 für sein Zimmer
		\item Die Summe der Zahlen soll 12 sein
		\item Alice mag keine ungeraden Zahlen
	\end{itemize}

	Findet alle 14 möglichen Kombinationen, die Zimmer zu nummerieren.
\end{frame}

\begin{frame}{Mathematiker-WG}
	\code{demos/mathematiker_wg.pl}
\end{frame}

\subsection{Detektivrätsel}

\begin{frame}{Detektivrätsel}
	\begin{figure}
		\includegraphics[width=0.8\textwidth]{images/victor}
	\end{figure}
\end{frame}

\begin{frame}{Detektivrätsel}
	Im Fall des Mordes an ihrem Nachbarn Victor sind nun Alice, Bob und Carl die einzigen Zeugen.
	\begin{itemize}
		\item Alice:
		\begin{itemize}
			\item Bob war mit dem Opfer befreundet.
			\item Carl und das Opfer waren verfeindet.
		\end{itemize}
		\item Bob:
		\begin{itemize}
			\item Ich war überhaupt nicht daheim!
			\item Ich kenne den garnicht!
		\end{itemize}
		\item Carl:
		\begin{itemize}
			\item Ich bin unschuldig!
			\item Wir waren zum Zeitpunkt der Tat alle in der WG.
		\end{itemize}
	\end{itemize}
\end{frame}

\begin{frame}{Detektivrätsel}
	\code{demos/detektiv.pl}
\end{frame}

\begin{frame}{Detektivrätsel}
	\code{demos/detektiv2.pl}
	
	\begin{itemize}
		\item Was macht \texttt{select/3}?
		\item \texttt{inkonsistent/1} nimmt Liste von Namen und prüft auf paarweisen Widerspruch der Aussagen
	\end{itemize}
\end{frame}

% aussage(alice, ...), aussage(bob, ...), jeweils freund, feind, fremder, daheim, unterwegs
% widerspruch
% select/3, oder selber implementieren
% inkonsistent(Liste), prüft ob paarweise widersprüche vorliegen

\subsection{Bettenverteilung}

\begin{frame}{Schlafplätze im Gefängnis}
	\begin{figure}
		\includegraphics[width=0.8\textwidth]{images/bett}
	\end{figure}
\end{frame}

\begin{frame}{Dinesman's multiple-dwelling problem}
	Bob kommt nun ins Gefängnis.
	Aaron, Bob, Connor, David und Edison müssen sich zu fünft ein sehr breites Bett teilen.

	\begin{itemize}
		\item Aaron will nicht am rechten Ende liegen
		\item Bob will nicht am linken Ende liegen
		\item Connor will an keinem der beiden Enden liegen
		\item David will weiter rechts liegen als Bob
		\item Connor schnarcht sehr laut;\\Bob und Edison sind sehr geräuschempfindlich
		\begin{itemize}
			\item $\leadsto$ Bob will nicht direkt neben Connor liegen
			\item $\leadsto$ Edison will nicht direkt neben Connor liegen
		\end{itemize}
	\end{itemize}

	Wie können die 5 Schlafplätze verteilt werden?
\end{frame}

\begin{frame}{Schlafplätze im Gefängnis}
	\code{demos/schlafplaetze.pl}

	\begin{itemize}
		\item Fügt weitere benötigte Tests ein
		\item Implementiert:
		\begin{itemize}
			\item \texttt{distinct/1} prüft Listenelemente auf paarweise Ungleichheit
			\item \texttt{adjacent/2} prüft, ob $|A - B| = 1$
		\end{itemize}
	\end{itemize}
\end{frame}

% ansatz: multipleDwelling(B, C, F, M, S) :- ....
% distinct/1
% adjacent/2

\begin{frame}{Putzdienst}
	\begin{figure}
		\includegraphics[width=0.7\textwidth]{images/putzdienst}
	\end{figure}
\end{frame}

\begin{frame}{Putzdienst}
	\begin{itemize}
		\item Auf Aaron, Bob, Connor, David und Edison sollen 4 Einheiten Putzdienst übernehmen
		\item Da sie sich nicht einigen können, wer faulenzen darf, wendet ein Wärter folgendes Vorgehen an:
		\begin{itemize}
			\item Die fünf werden im Kreis aufgestellt
			\item Der Wärter stellt sich in die Mitte
			\item Beginnend bei 12 Uhr dreht er sich im Uhrzeigersinn und teilt jeden $k$-ten (bspw. $k = 2$) Insassen zum Putzdienst ein
			\begin{itemize}
				\item D.h. es werden immer $k - 1$ Insassen übersprungen
			\end{itemize}
		\end{itemize}
	\end{itemize}

	An welcher Stelle muss Bob stehen, um nicht putzen zu müssen?\\
	\pause
	An welcher Stelle muss Bob bei 41 Insassen und $k = 3$ stehen?
\end{frame}

\begin{frame}{Putzdienst}
	\code{demos/putzdienst.pl}

	\begin{itemize}
		\item Weiter Fälle für \texttt{helper/4}:
		\begin{itemize}
			\item \texttt{C = 0} $\leadsto$ Element entfernen
			\item Ansonsten: Element hinten wieder anhängen
		\end{itemize}
	\end{itemize}
\end{frame}

\subsection{Quellen}

\begin{frame}{Quellen der Aufgaben}
	Zum Nachlesen und Vergleichen mit Lösungen in anderen Programmiersprachen:
	\begin{itemize}
		\item WG --- \href{https://rosettacode.org/wiki/Department_Numbers}{Rosetta Code: Department Numbers}
		\item Detektiv --- \href{https://github.com/Anniepoo/prolog-examples/blob/master/newdetective.pl}{Anniepoo/prolog-examples}
		\item Schlafplätze --- SICP, S. 418
		\item Putzdient --- \href{https://rosettacode.org/wiki/Josephus_problem}{Rosetta Code: Josephus problem}
	\end{itemize}
\end{frame}

% quellen:
% multiple dwelling: SICP S. 418
% detective: https://github.com/Anniepoo/prolog-examples/blob/master/newdetective.pl
% department numbers: https://rosettacode.org/wiki/Department_Numbers
% josephus problem: https://rosettacode.org/wiki/Josephus_problem#Python

\end{document}
