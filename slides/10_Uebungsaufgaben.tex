\documentclass{beamer}
\usetheme{metropolis}

\usepackage[ngerman]{babel}
\usepackage[autostyle=true,german=quotes]{csquotes}
\usepackage[linewidth=1pt]{mdframed}
\usepackage{hyperref}
\usepackage{makecell}
\usepackage{pifont}
\usepackage{tikz}
\usetikzlibrary{positioning, calc, arrows, fit, decorations.pathreplacing, shapes, shapes.multipart, snakes}
\usepackage{verbatim}
\usepackage{tabularx}
\usepackage{textcomp}
\usepackage{centernot}
\usepackage{amsmath}
\usepackage{xcolor}
\usepackage{tikz}
%\usepackage{pdfpages}

\batchmode

\hypersetup{
	colorlinks,
	urlcolor=blue,
	linkcolor=black % for ToC
}
\newenvironment{qaa}[1]{
	#1

	\begin{mdframed}
		\small
}{
	\end{mdframed}
}

\newcommand{\true}{\ding{51}}
\newcommand{\false}{\ding{55}}
\newcommand{\code}[1]{
	\begin{mdframed}
		\verbatiminput{#1}
	\end{mdframed}
}

\title{Tutorium 10: Übungsaufgaben}
% \subtitle{}
\author{Paul Brinkmeier}
\institute{Tutorium Programmierparadigmen am KIT}
\date{8. Januar 2020}

\begin{document}

\begin{frame}
	\titlepage
\end{frame}

\section{Heutiges Programm}

\begin{frame}{Programm}
	\begin{itemize}
		\item Wiederholung Typinferenz
		\item Typinferenz mit \textsc{Let}
		\item Übungsaufgaben
	\end{itemize}
\end{frame}

\section{Typinferenz}

\newcommand{\aeq}{\stackrel{\alpha}{=}}
\newcommand{\naeq}{\stackrel{\alpha}{\neq}}
\newcommand{\eeq}{\stackrel{\eta}{=}}

\newcommand{\E}{\;}

\newcommand{\liin}[2]{#1\E{}#2}
\newcommand{\liiin}[3]{#1\E{}#2\E{}#3}
\newcommand{\livn}[4]{#1\E{}#2\E{}#3\E{}#4}
\newcommand{\lvn}[5]{#1\E{}#2\E{}#3\E{}#4\E{}#5}

\newcommand{\lii}[2]{(#1\E{}#2)}
\newcommand{\liii}[3]{(#1\E{}#2\E{}#3)}

\newcommand{\liir}[2]{\textcolor{red}{\underline{(}}#1\E{}#2\textcolor{red}{\underline{)}}}
\newcommand{\liiir}[3]{\textcolor{red}{\underline{(}}#1\E{}#2\E{}#3\textcolor{red}{\underline{)}}}

\newcommand{\subst}[3]{(#1)\left[#2\,\to\,#3\right]}
\newcommand{\abs}[2]{\lambda{}#1.#2}
\newcommand{\reducesTo}[1]{\stackrel{#1}{\implies}}
\newcommand{\tikzmark}[3]{\tikz[baseline, remember picture]{
	\node[fill=#1,draw] (#2) {#3};
}}

\newcommand{\typeRule}[3]{\frac{#2}{#3} \textrm{\textsc{#1}}}

\begin{frame}{Wiederholung: Typinferenz}
	Vorgehensweise zur Typinferenz:
	\begin{itemize}
		\item Stelle Typherleitungsbaum auf
		\begin{itemize}
			\item In jedem Schritt werden neue Typvariablen $\alpha_i$ angelegt
			\item Statt die Typen direkt im Baum einzutragen, werden Gleichungen in einem Constraint-System eingetragen
		\end{itemize}
		\item Unifiziere Constraint-System zu einem Unifikator
		\begin{itemize}
			\item Robinson-Algorithmus, im Grunde wie bei Prolog
		\end{itemize}
	\end{itemize}
\end{frame}

\begin{frame}{Robinson-Algorithmus}
	\code{code/robinson-l.pseudo}

	Erzeugt Unifikator zu einem Constraint-System.
\end{frame}

\begin{frame}{Typinferenz: Übungsaufgabe}
	\only<1>{
	\begin{equation*}
		\typeRule{Abs}{
			...
		}{
			\texttt{f} : \textrm{int} \to \beta \vdash \abs{\texttt{x}}{\liin{\texttt{f}}{\texttt{x}}} : \alpha_1
		}
	\end{equation*}

	\begin{itemize}
		\item \enquote{Finde den allgemeinsten Typen $\alpha_1$ von $\abs{\texttt{x}}{\liin{\texttt{f}}{\texttt{x}}}$}
	\end{itemize}
	}

	Erinnerung:

	\begin{itemize}
		\item Baum mit durchnummerierten $\alpha_i$ aufstellen
		\item Constraints sammeln:
	\end{itemize}

	\begin{columns}
		\scriptsize
		\begin{column}{0.3\textwidth}
			\begin{equation*}
			\typeRule{Var}{
				\Gamma(x) = \sigma
				\;\;\;\;
				\sigma \succeq \iota
			}{
				\Gamma \vdash x : \tau
			}
			\end{equation*}

			Constraint: $\{ \iota = \tau \}$
		\end{column}
		\begin{column}{0.3\textwidth}
			\begin{equation*}
			\typeRule{App}{
				\Gamma \vdash f : \xi
				\;\;\;\;
				\Gamma \vdash x : \phi
			}{
				\Gamma \vdash \liin{f}{x} : \alpha
			}
			\end{equation*}

			Constraint: $\{ \xi = \phi \to \alpha \}$
		\end{column}
		\begin{column}{0.3\textwidth}
			\begin{equation*}
			\typeRule{Abs}{
				\Gamma, p : \pi \vdash b : \beta
			}{
				\Gamma \vdash \abs{p}{b} : \alpha
			}
			\end{equation*}

			Constraint: $\{ \alpha = \pi \to \beta \}$
		\end{column}
	\end{columns}

	\begin{itemize}
		\item Constraint-System auflösen (Robinson-Algorithmus)
	\end{itemize}
\end{frame}

\newcommand{\unifierLet}{\sigma{}_{\textrm{\textsc{Let}}}}

\begin{frame}{Wozu brauchen wir \textsc{Let}?}
	\begin{itemize}
		\item Findet den allgemeinsten Typen von $\alpha_1$ von $(\abs{\texttt{x}}{\liin{\texttt{x}}{\texttt{x}}})\;\abs{\texttt{y}}{\texttt{y}}$
		\pause
		\item $\leadsto$ Geht nicht, $\abs{\texttt{y}}{\texttt{y}}$ hat eine feste Struktur, kann nicht auf sich selbst angewendet werden
		\item Wir müssten \texttt{x} einen Typen geben, den man mehrmals verwenden kann: $\texttt{x} : \forall \alpha . \alpha \to \alpha$
		\pause
		\item Dafür gibt es \textsc{Let}:
	\end{itemize}

		\begin{equation*}
		\typeRule{Let}{
			\Gamma \vdash y : \pi
			\;\;\;\;
			\Gamma' \vdash b : \beta
		}{
			\Gamma \vdash \textrm{let}\;x = y\;\textrm{in}\;b : \tau
		}
		\end{equation*}
	
	\begin{itemize}
		\item Mit den Constraints ist das hier etwas komplizierter:
		\begin{itemize}
			\item Finde Unifikator $\unifierLet$ und allg. Typ $\pi$ für $y$
			\item $\Gamma' = \unifierLet(\Gamma)\;,\;x : ta(\unifierLet(\pi), \unifierLet(\Gamma))$
			\item $ta(\tau, \Gamma)$ bindet alle in $\Gamma$ freien Typvariablen mit einem $\forall$ in $\tau$
			\item Bspw. $ta(\alpha \to \beta, x : \beta, y : \delta) = \forall \alpha . \alpha \to \textrm{int}$
		\end{itemize}
	\end{itemize}
\end{frame}

\end{document}
