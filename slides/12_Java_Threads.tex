\documentclass{beamer}
\usetheme{metropolis}

\usepackage[ngerman]{babel}
\usepackage[autostyle=true,german=quotes]{csquotes}
\usepackage[linewidth=1pt]{mdframed}
\usepackage{hyperref}
\usepackage{makecell}
\usepackage{pifont}
\usepackage{tikz}
\usetikzlibrary{positioning, calc, arrows, fit, decorations.pathreplacing, shapes, shapes.multipart, snakes}
\usepackage{verbatim}
\usepackage{tabularx}
\usepackage{textcomp}
\usepackage{centernot}
\usepackage{amsmath}
\usepackage{xcolor}
\usepackage{tikz}
\usepackage{underscore}

\batchmode

\hypersetup{
	colorlinks,
	urlcolor=blue,
	linkcolor=black % for ToC
}
\newenvironment{qaa}[1]{
	#1

	\begin{mdframed}
		\small
}{
	\end{mdframed}
}

\newcommand{\true}{\ding{51}}
\newcommand{\false}{\ding{55}}
\newcommand{\code}[1]{
	\begin{mdframed}
		\verbatiminput{#1}
	\end{mdframed}
}

\title{Tutorium 12: Parallelprogrammierung in Java}
% \subtitle{}
\author{Paul Brinkmeier}
\institute{Tutorium Programmierparadigmen am KIT}
\date{20. Januar 2020}

\begin{document}

\begin{frame}
	\titlepage
\end{frame}

{
	\usebackgroundtemplate{\includegraphics[width=\paperwidth]{EulenfestWerbung}}
	\begin{frame}[plain]
	\end{frame}
}

\begin{frame}{Parallelprogrammierung}
	ProPa-Stoff zu Parallelprogrammierung:

	\begin{itemize}
		\item Grundlegende Begriffe
		\item Message Passing, wurde in OS \emph{kurz} behandelt (\enquote{message queues})
		\item Shared Memory + Synchronisierung, wie in SWT1, OS, etc.
		\begin{itemize}
			\item In Java, mit ein paar Details zur JVM
		\end{itemize}
	\end{itemize}
\end{frame}

\section{Parallelprogrammierung in Java}

\begin{frame}{Übungsaufgabe: Dateiverarbeitung}
	\begin{itemize}
		\item Klasse \texttt{Dirwalker} in \texttt{demos/java/dirwalker}
		\item Findet Größe eines Verzeichnisbaumes
		\pause
		\item Konstruktor bekommt 1 Argument:
		\begin{itemize}
			\item \texttt{File root}: Ordner, in dem die Suche beginnt
		\end{itemize}
		\item \texttt{run()} startet Durchsuchen von \texttt{root}
		\pause
		\item Bisher: rekursive, nicht-parallele Implementierung
		\item Aufgabe: Parallelisiert \texttt{run()}, soweit es geht!
		\begin{itemize}
			\item Einmal mit \texttt{Thread}s machen
			\item Dann mit \texttt{ExecutorService}, etwas fortgeschrittener
			\item Vergleichen!
		\end{itemize}
	\end{itemize}
\end{frame}

\section{Ende}

\begin{frame}{Ende}
	\begin{itemize}
		\item Im Campus-System kann man sich bis zum 17.03. für die ProPa-Klausur anmelden
		\item \href{https://campus.studium.kit.edu/renewal/payment.php}{Rückmelden} bis zum 15.02.
		\item Eulenfest am 23.01.!
	\end{itemize}
\end{frame}

\end{document}
