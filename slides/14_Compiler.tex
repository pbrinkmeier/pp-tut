\documentclass{beamer}
\usetheme{metropolis}

\usepackage[ngerman]{babel}
\usepackage[autostyle=true,german=quotes]{csquotes}
\usepackage[linewidth=1pt]{mdframed}
\usepackage{hyperref}
\usepackage{makecell}
\usepackage{pifont}
\usepackage{tikz}
\usetikzlibrary{positioning, calc, arrows, fit, decorations.pathreplacing, shapes, shapes.multipart, snakes}
\usepackage{verbatim}
\usepackage{tabularx}
\usepackage{textcomp}
\usepackage{centernot}
\usepackage{amsmath}
\usepackage{xcolor}
\usepackage{tikz}
\usepackage{underscore}

\batchmode

\hypersetup{
	colorlinks,
	urlcolor=blue,
	linkcolor=black % for ToC
}
\newenvironment{qaa}[1]{
	#1

	\begin{mdframed}
		\small
}{
	\end{mdframed}
}

\newcommand{\true}{\ding{51}}
\newcommand{\false}{\ding{55}}
\newcommand{\code}[1]{
	\begin{mdframed}
		\verbatiminput{#1}
	\end{mdframed}
}

\title{Tutorium 14: Compiler}
% \subtitle{}
\author{Paul Brinkmeier}
\institute{Tutorium Programmierparadigmen am KIT}
\date{03. Februar 2020}

\begin{document}

\begin{frame}
	\titlepage
\end{frame}

\section{Einführung}

\begin{frame}{Compiler in ProPa}
	\begin{itemize}
		\item Ein bisschen...
		\begin{itemize}
			\item Lexikalische Analyse (Tokenisierung)
			\item Syntaktische Analyse (Parsen)
			\item \textcolor{gray}{Semantische Analyse (Optimierung)}
			\item Codegenerierung
		\end{itemize}
		\pause
		\item Klausur:
		\begin{itemize}
			\item SLL(k)-Form beweisen
			\item Rekursiven Abstiegsparser schreiben/vervollständigen
			\item First/Follow-Mengen berechnen
			\item Java-Bytecode
				% Zeigen: SGML-Aufgabe (19WS)
				% Zeigen: Java-BC-Aufgabe (16SS), Code-Generierung nächste Mittwoch und Freitag (findet noch statt)
		\end{itemize}
	\end{itemize}
\end{frame}

% Aufgaben:
% Nicht-SLL(1)-Teilmenge von JSON-Subset ist SLL(n) für n \in 1, 2?
% Rekursiver Abstieg für JSON-Subset
% Rekursiver Abstieg für Klausuraufgabenliste
% Java Bytecode

\section{Lexikalische Analyse}

\section{Syntaktische Analyse}

\section{Semantische Analyse}

\section{Codegenerierung}

% Klausurtermin, Rückmeldung, Cheatsheet, Aufgaben.md, etc.

\end{document}
